TODO comment faire les defintions pour que ca ne souit pas du plagiat????
\begin{definition}[$l^2$]
    
\end{definition}

\begin{definition}[Produit scalaire]
    Le produit scalaire dun $\mathbb{C}\mathrm{-espace}$ vectoriel $V$ est une application bilinear conjuge-symetrique. $$\langle ., . \rangle : V \times V \rightarrow \mathbb{C}$$

    telle que pour $ \forall x, y, z\in V$ et pour $\forall \lambda \in \mathbb{C}$,
    \begin{itemize}
        \item $\langle x, y \rangle = \overline{\langle y,x \rangle}$
        \item $\langle \lambda x, y \rangle = \lambda\langle x,y \rangle$
        \item $\langle x+y, z \rangle = \langle x,z \rangle + \langle y,z \rangle$
        \item $\langle x,x \rangle > 0 $ alors $x \neq 0 $
    \end{itemize}
    
    De cela on peut aussi definir l'espace préhilbertien qui est le pair $\langle V, \langle.,.\rangle \rangle$ ou $V$ est un $\mathbb{C}\mathrm{-espace}$ vectoriel et $\langle.,.\rangle$ est le produit scalaire sur $V$.
\end{definition}

TODO est que cela est vraie tout le temps ou dans les espaces fini pu $\mathbb{C}^\mathbb{N}$ Car si cest pas toujoursle cas TODO definir en espace infini

Si $x, y \in \mathbb{C}^n $ alors,
$$\langle x,y \rangle = \sum_{i=1}^{n} x_i \overline{y_i}$$
ou $x = (x_1, \ldots, x_n), y = (y_1, \ldots, y_n)$ et $\overline{y_i}$ est le conjuge complexe de $y_i$.  \newline (Petit rappel: si $z \in \mathbb{C} \Rightarrow \overline{z} = \mathrm{Re}(z) - \mathrm{Im}(z) $)

TODO je suis pas sur si cet defini seulement si x et y sont de mem dimension ou pas 
\begin{definition}[Espace dual]
    
\end{definition}



\begin{definition}[Espace d'Hilbert]

\end{definition}

\begin{definition}[dérivée de Gâteaux ou dérivée directionnelle]
\end{definition}
    
\begin{definition}[Dérivée au sens de Fréchet]
    TODO Soit $E$ un espace de Hilbert. On dit que f est Fréchet différentiable en x
    s'il existe $p \in E$ tel que:
    \begin{equation}
        f(x+h) = f(x)+ \langle p, h \rangle +O(h^2) 
    \end{equation}
    On dit que p est la derivee (ou la differentielle ou le gradient) de $f$ en $x$ si ca existe peux etre note comme ci dessous. (Tout les ecriture ci dessous sont equivaslentes) % TODO metre dans la biblio livre de Philippe G. Ciarlet 
    Donc la formule de Taylor-Young pour les fonctions $f: \mathbb{R}^n\rightarrow\mathbb{R}$ de classe $\mathcal{C} ^2$ est: 
    \begin{equation}
        f(x+h) = f(x)+ f'(x) h  + \frac{1}{2} f''(x)\langle h, h \rangle +O(||h||^3_2)
    \end{equation}
    \begin{equation}
        f(x+h) = f(x)+ \langle \nabla f(x), h \rangle + \frac{1}{2} \langle \nabla^2 f(x)\cdot h, h \rangle +O(\langle h, \langle h, h\rangle \rangle)
    \end{equation}
    \begin{equation}
        f(x+h) = f(x)+ (\nabla f(x))^T h \rangle + \frac{1}{2} h^T \nabla^2 f(x) h +O(h^T h h)
    \end{equation}
    % TODO est que les restes sont bon? Non mais siont les quelles les bons ???

\end{definition}